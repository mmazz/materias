%\documentclass[handout]{beamer}
\documentclass[10pt]{beamer}
%for(int i=0; i<n; i++){\documentclass[10pt,handout]{beamer}
\usepackage[spanish]{babel}
% % \usepackage[backend=biber, style=authoryear-icomp]{biblatex}
\resetcounteronoverlays{exx}
\usepackage{mdframed}
\usepackage{tikz}
\usepackage{blindtext}
\usepackage{tipa}
% \usepackage{cgloss4e}
% \usepackage{gb4e}
% \usepackage{qtree}
\usepackage{cancel}
\usepackage{wrapfig}
\usepackage{soul}
\usepackage{enumerate}
\usepackage{longtable}
\graphicspath{ {./figures/} } % declaramos donde estan las imagenes
\usepackage[labelformat=simple]{subcaption} % para varias imagenes juntas
\renewcommand\thesubfigure{(\alph{subfigure})}
\usepackage[utf8]{inputenc}
\usepackage{amsmath}
\usepackage{amsfonts} % simbolos como el I de matriz identidad
\usepackage{bm}
\usepackage{graphicx} % paquete para ver imagenes
\usepackage{setspace}
\usepackage[T1]{fontenc}
\usepackage{parskip}
\usepackage{color}
\usepackage{framed}

\usetheme{Copenhagen}
\definecolor{frenchblue}{rgb}{0.0, 0.45, 0.73} % ESTE!!!!

\setbeamercolor{block body}{bg=frenchblue!50}
\setbeamercolor*{structure}{fg=frenchblue,bg=blue}
\setbeamercolor{normal text}{fg=black}
\setbeamercolor{frametitle}{bg=black}
\setbeamertemplate{frametitle}[default][center]
\setlength{\parskip}{12pt}
\useoutertheme{infolines} % me comia mucho espacio de la otra fgorma
\makeatother
\setbeamertemplate{footline}
{
  \leavevmode%
  \hbox{%
  \begin{beamercolorbox}[wd=.3\paperwidth,ht=2.25ex,dp=1ex,center]{author in head/foot}%
    \usebeamerfont{author in head/foot}\insertshortauthor
  \end{beamercolorbox}%
  \begin{beamercolorbox}[wd=.6\paperwidth,ht=2.25ex,dp=1ex,center]{title in head/foot}%
    \usebeamerfont{title in head/foot}\insertshorttitle
  \end{beamercolorbox}%
  \begin{beamercolorbox}[wd=.1\paperwidth,ht=2.25ex,dp=1ex,center]{date in head/foot}%
    \insertframenumber{} / \inserttotalframenumber\hspace*{1ex}
  \end{beamercolorbox}}%
  \vskip0pt%
}
\makeatletter
\setbeamertemplate{navigation symbols}{}
%\setbeameroption{show notes}
\setbeameroption{hide notes}
\renewcommand{\CancelColor}{\color{red}}

\usepackage{hyperref}

\title[RISK II]{Presentación RISC II}
\author[Matias Mazzanti]{Matias Mazzanti}


\institute{DC-UBA}
\date{05 de Septiembre de 2022}

\titlegraphic{\includegraphics[,height=2cm,keepaspectratio]{../logo.pdf}     }
%\logo{\includegraphics[height=2.5cm]{logo.PDF}}

\begin{document}

\begin{frame}

\maketitle

\end{frame}

% cuál es tu background que estuviste haciendo y qué querés hacer
% Con quién vas a trabajar en los planes
% un poquito de cuál es la idea del doctorado aunque todavía este verde

\section{Presentaci\'on}
\begin{frame}
\frametitle{Introducción}

\begin{itemize}
  \item Licenciado en Ciencias de la Fisica.
  \item Alumno doctoral de Esteban Mocskos: Laboratorio Indisciplinario de Computación de Alto Rendimiento. UBA.
  \item Tema doctoral: Aceleración por Hardware.
\end{itemize}

\end{frame}



\begin{frame}
\frametitle{Introducción}
\begin{figure}[h!]
    \centering
    \includegraphics[scale=0.3]{road.jpg}
\end{figure}
\end{frame}


\begin{frame}
\frametitle{Origenes y tesis}

 Transición a computación desde Lic Fisica: optativas a la computación.
\begin{columns}
    \column{0.5\textwidth}
\vspace{0.5cm}

Tesis dentro del departamento de computación de la UBA.
\vspace{0.5cm}

Estimaciones Bayesianas de la habilidad en jugadores del juego de mesa de Go.

\vspace{0.5cm}
Se propuso una mejora en el sistema de rankeo para plataformas online.
    \column{0.5\textwidth}
\begin{figure}[h!]
    \centering
    \includegraphics[scale=2.]{go.jpg}
\end{figure}
\end{columns}
\end{frame}

\begin{frame}
\frametitle{Doctorado}
Propuesta original: Continuar la tesis utilizando el modelo de estimación y buscar su paralelización mediante GPU.

La paralelización para el modelo no se justificaba $\rightarrow$ utilizar el modelo para responder otras preguntas.

Modelar y estudiar el aprendizaje humano.

El camino que tomó el doctorado en su primer año no era lo pensado $\to$ cambio de planes.



\end{frame}

\begin{frame}
\frametitle{Cambio Doctorado}


Director: Esteban Mocskos UBA

Codirector: Augusto Vega IBM

Fully Homomorphic Encryption (FHE).

Operar con datos encriptados.

Lograr aplicar métodos de machine learning sin necesidad de desencriptar la data.

Muy demandado por el extenso uso de nubes.

Actualidad: muy lento para uso real.

\begin{mdframed}[backgroundcolor=frenchblue!20]\centering
  Acelerar mediante hardware.
  \end{mdframed}

\end{frame}
\begin{frame}
\frametitle{Posibilidades}
Aceleracion ?
\end{frame}


\begin{frame}
\frametitle{}
  \begin{figure}[h!]
      \centering
      \includegraphics[scale=0.3]{agradecimientos.jpg}
  \end{figure}
\end{frame}
\end{document}
